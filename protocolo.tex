\documentclass[12pt]{report}
\usepackage[utf8]{inputenc}
\usepackage[spanish]{babel}
\begin{document}
\title{ {Análisis de tiempos de lanzamiento de proyectos comunitarios
    de software libre mediante análisis de redes sociales: Caso
    practico Debian GNU/Linux}\\ {Protocolo} } \author{Wences Arana}
\maketitle

\tableofcontents
\chapter*{Propósito del proyecto}

Todo software desde  su comienzo como una idea general  hasta llegar a
ser un sistema  integro y funcional, conlleva un  proceso ordenado con
limites  definidos   en  cuestiones   de  tiempo  y   cumplimiento  de
metas.  Dicho  cumplimiento de  metas  debe  ser  dentro de  un  marco
tolerable de tiempo, esto con el  fin de tener un proceso eficiente de
desarrollo  de software,  un  mejor manejo  de  costos de  producción,
calidad del mismo  entre otras cualidades. Por esto  mismo la medición
de los factores que influyen en el atraso del cumplimiento de metas se
hace  de  vital  importancia  para  el estudio  de  la  ingeniería  de
software,  para  lograr  crear  un  proceso  eficiente  de  desarrollo
replicable y  dentro de  los recursos disponibles.  Por esto  mismo se
plantea realizar  dicho análisis  utilizando el área  de la  teoría de
análisis de redes  sociales con el fin  de por medio de  un estudio de
los  datos disponibles  crear  un modelo  replicable  en proyectos  de
software con  características similares,  en nuestro  caso se  hará el
estudio en base  a sistemas operativos libres  desarrollados de manera
colaborativa por  voluntarios y otros actores  interesados, tomando en
cuenta  factores  relacionados  con   la  unidad  mínima  de  software
denominada  paquete  que conforma  una  red  de  estos  que a  su  vez
conforman el  sistema operativo en  su totalidad, además de  las redes
que  existen en  las relaciones  sociales de  los participantes  en el
desarrollo  de  dicho  software.  El   estudio  servirá  a  todos  los
interesados en obtener un modelo  para poder encontrar posibles fallas
en la gestión del proyecto  relacionadas con los distintos componentes
que forman parte de su proyecto de software tanto social como técnico,
tangible e intangible.

\chapter*{Introducción}
La  teoría  de  análisis  de   redes  tiene  como  objetivo  encontrar
relaciones en eventos de la naturaleza que no parecen tener incidencia
unos con  otros a primera vista,  sin embargo al realizar  un análisis
profundo de sus relaciones, en algunos casos, pueden encontrarse desde
relaciones de causa efecto entre  eventos, actores, recursos, hasta la
importancia  de algunos  de  estos  en el  funcionamiento  de todo  el
sistema   en   su   conjunto,   subgrupos   relacionados,   relaciones
inesperadas, entre otras. Utilizando dicha teoría se pretende analizar
el  proyecto  colaborativo  de  software libre  Debian  y  su  sistema
operativo Debian GNU/Linux  con el objetivo de  encontrar los factores
que  inciden en  el tiempo  de  lanzamiento de  versiones estables  de
este.  Para dicho  objetivo se  manejaran varios  enfoques para  poder
medir dicho  impacto y estos  son: utilizar como principales  nodos la
unidad mínima funcional de software en el proyecto denominado paquete,
la figura  de desarrollador de software  siendo en este caso  tanto un
desarrollador Debian  como un desarrollador original  del software que
da  origen al  paquete,  como  relaciones entre  nodos  se tomaran  en
cuenta, dependencias  de construcción entre paquetes,  dependencias en
tiempo de ejecución  de los paquetes, bugs detectados  en los paquetes
que  interfieran  en  la  relación   entre  ellos,  cantidad  de  bugs
reportados  de parte  de un  desarrollador Debian  a un  desarrollador
upstrem, cantidad de parches aplicados  por parte de upstream enviados
por   un    desarrollador   debian,   cantidad   de    correos   entre
desarolladores. Con  esta informacion  modelada se podra  analizar por
medio  de  algoritmos  existentes  de  teoría  de  análisis  de  redes
sociales, en caso  de poder ser aplicables, de lo  contrario se creara
el adecuado  para poder realizar  de manera  apropiada los datos  y de
esta  forma poder  encontrar, patrones  de centralidad,  incidencia de
clanes,  nodos  de  mayor  importancia  etcetera,  y  con  esto  podrá
interpretarse los valores y hacer un análisis del impacto en el tiempo
de desarrollo de una nueva versión  estable del proyecto de software a
otra.

\chapter*{Antecedentes}
En el campo de estudio de la teoria de analisis de redes, se han
realizado investigaciones realcionadas a proyectos de software
pudiendo mencionar analisis de comunicacion entre desarrolladores en
un proyecto de software\footnote{Crowston, Kevin, and James Howison.
  ``The social structure of free and open source software
  development.''  First Monday 10.2 (2005).}, las implicaciones de la
cordinacion de la arquitectura de software en proyectos de
software\footnote{Ryan, Sharon, and Rory V. O\'Connor. ``Development
  of a team measure for tacit knowledge in software development
  teams.'' Journal of Systems and Software 82.2 (2009): 229-240.},
para encontrar patrones de colaboracion en equipos de desarrollo de
software\footnote{Ehrlich, Kate, Giuseppe Valetto, and Mary
  Helander. ``Seeing inside: Using social network analysis to
  understand patterns of collaboration and coordination in global
  software teams.'' Global Software Engineering, 2007. ICGSE
  2007. Second IEEE International Conference on. IEEE, 2007.}, entre
otros casos de estudio en el area de ingenieria de software. De tal
forma que el estudio se encargara en el area de tiempos de lanzamiento
y de desarrollo que se basan en dichos estudios anteriores,
aplicandolo a un proyecto colaborativo de software libre y definiendo
un modelo en base a los datos, buscando encontrar los principales
cuellos de botella y areas a mejorar para poder eficientar el proceso
de desarrollo de software de la distribucion Debian GNU/Linux.

\chapter*{Justificacion}
La investigacion tiene un fundamento esencial de existir debido a la
necesidad de poder encontrar los problemas que se llevan a acabo en
proyectos colaborativos de software de gran envergadura.

El motivo de la eleccion de un proyecto comunitario de software libre
como Debian Gnu/Linux se debe a factores sociales y tecnicos, sociales
debido a que la estructura organizacional de la comunidad de
desarrolladores es publica y abierta lo cual es una ventaja para poder
obtener informacion de relaciones entre toda la comunidad. Desde el punto de
vista tecnico, al ser un proyecto que desarrolla un sistema operativo
de software libre permite poder tener acceso a toda la informacion
referente a los enlaces y relaciones que existen entre las unidades de
software que forman al sistema operativo en general, desde relaciones
de dependencias directas hasta relaciones indirectas en el
funcionamiento integral del sistema operativo como tal.

Debido a lo anteriormente expuesto tomando en cuenta otras teorias de
analisis de datos existentes, la mas adecuada para analizar
informacion y poder encontrar factores medibles que afecten a la
liberacion de nuevas versiones de software es la teoria de analisis de
redes sociales esto debido a que la organizacion del proyecto tanto
social como tecnicamente pueden ser descritos como una red donde
pueden encontrarse tendencias y poder aplicarse un modelo replicable
en otras investigaciones, el proyecto ya provee un entramado
interrelacionado lo cual facilita la utilizacion de analisis de redes
sociales en la organizacion

El proyecto es beneficioso por que brindara un modelo replicable para
todo gestor de software que cuente con un proyecto de software con
multiples componentes interrelacionados y un equipo de colaboradores
heterogeneos tanto internos(desarrolladores) como externos(terceros).

Las implicaciones que tendra el trabajo ira orientado a la creacion de
un modelo replicable para determinar tiempos de lanzamientos.

El trabajo va dirigido a toda persona que necesite modelos para poder
determinar que puntos modificar en su organizacion de manera tecnica y
social de tal forma que pueda mejorar tiempos de lanzamiento y
obteniendo calidad en un tiempo razonable con una mayor anticipacion.

\chapter*{MARCO TEÓRICO}

En el caso del protocolo, el marco teórico funge como un marco
conceptual, indicando ciertas definiciones conceptuales y
explicaciones breves con que se basará el proyecto de
investigación. Para la entrega del protocolo esta sección debe tener
aproximadamente de 2 a 3 páginas de conceptualización.

\end{document}

%TODO referencia http://www.researchgate.net/publication/271625890_Analysis_of_activity_in_open-source_communities_using_social_network_analysis_techniques
%TODO http://www3.nd.edu/~oss/Papers/HICSS40_final.pdf
