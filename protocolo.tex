\documentclass[12pt]{report}
\usepackage[utf8]{inputenc}
\usepackage[spanish]{babel}
\usepackage{hyperref}

%parafernalia para las revisiones
\usepackage{verbatim}
\usepackage{xargs}
\usepackage{changebar}
\usepackage[colorinlistoftodos, textwidth=65mm, shadow]{todonotes}
\usepackage[paperwidth=275.9mm, paperheight=279.4mm]{geometry}
\setlength{\oddsidemargin}{35mm}
\setlength{\evensidemargin}{35mm}
\setlength{\voffset}{-1in}
\setlength{\hoffset}{-1in}
\setlength{\textwidth}{156mm}
\setlength{\topmargin}{4mm}
\setlength{\headheight}{10mm}
\setlength{\headsep}{12mm}
\setlength{\topskip}{0mm}
\setlength{\textheight}{228mm}
%fin de parafernalia

\begin{document}  
\title{  
	{Análisis  de  tiempos  de  lanzamiento  de
	proyectos comunitarios  de software  libre mediante 
	análisis  de redes sociales:   Caso   practico    
	Debian   GNU/Linux}
	\\   {Protocolo}   
}
\author{Wences Arana}
\maketitle

\tableofcontents
\chapter*{Propósito del proyecto}

Todo software desde  su comienzo \todo{Inicia?}
como una idea general  hasta llegar a
ser un sistema  integro y funcional, conlleva un  proceso ordenado con
limites definidos  en cuestiones  de tiempo  y cumplimiento  de metas.
\todo{Esta es una afirmación muy fuerte para este estudio
considerando que Debian stable en realidad es un 
proceso organico sin limites, excepto el \textit{cuando esté listo},
y es una caracteristica inherente a Debian GNU/Linux}
Dicho cumplimiento de  metas debe ser dentro de un  marco tolerable de
tiempo, esto con el fin de tener un proceso eficiente de desarrollo de
software, un mejor  manejo de costos de producción,  calidad del mismo
entre otras cualidades.
\\
Por esto mismo la medición de los factores que
influyen  en el  atraso del  cumplimiento de  metas se  hace de  vital
importancia para el estudio de  la ingeniería de software, para lograr
crear un  proceso eficiente de  desarrollo replicable y dentro  de los
recursos  disponibles.   Por  esto  mismo \todo{"Por esto mismo" 
se repite demasiadas veces} 
se  plantea  realizar  dicho
análisis utilizando el área de la teoría de análisis de redes sociales
con el fin de  por medio de un estudio de  los datos disponibles crear
un  modelo replicable  en  proyectos de  software con  características
similares,  en nuestro  caso se  hará el  estudio en  base a  sistemas
operativos libres desarrollados de manera colaborativa por voluntarios
y otros  actores interesados, tomando en  cuenta factores relacionados
con la unidad  mínima de software denominada paquete  que conforma una
red  de estos  que  a su  vez  conforman el  sistema  operativo en  su
totalidad, además de las redes  que existen en las relaciones sociales
de los participantes  en el desarrollo de dicho  software.  El estudio
servirá  a todos  los  interesados  en obtener  un  modelo para  poder
encontrar posibles fallas en la  gestión del proyecto relacionadas con
los distintos componentes que forman  parte de su proyecto de software
tanto social como técnico, tangible e intangible.
\todo{La idea general del parrafo se entiende pero abusas de los
conectores, especialmente de "por esto", ver \url{http://mimosa.pntic.mec.es/ajuan3/lengua/l_conect.htm}}
\todo{El "proposito" del estudio es demostrar la necesidad de tu 
estudio, podes omitir la parte de SNA aca, de otra forma vas a tener
que explicarlo y quedaria redundante con la introducción}
\todo{Al lector puede no quedarle claro ¿porque es importante Debian?
, ¿Porque es importante para el mundo lanzamientos más predecibles?}

\begin{comment}
Tenes que poner en tu proposito cual es TU problema,
y porque es un problema importante para otros (individuos, comunidad)
, actualmente lo planteas como "el software que se atrasa es malo y
de panzaso creo que encontrare la solución mediante Debian Linux.
\end{comment}

\chapter*{Introducción}  
\todo{Debes iniciar por explicar el problema, no la técnica con la 
que pretendes resolverlo, la tecnica solo la debes justificar una vez
que ya se explicó el problema a detalle
-e.g. se pretende usar SNA por las caracteristicas organicas x,y,z,w
del proyecto-}
La teoría  de  análisis de  redes tiene  como
objetivo  encontrar relaciones  en  eventos de  la  naturaleza que  no
parecen tener incidencia  unos con otros a primera  vista, sin embargo
al realizar un análisis profundo  de sus relaciones, en algunos casos,
pueden  encontrarse desde  relaciones de  causa efecto  entre eventos,
actores,  recursos, hasta  la importancia  de algunos  de estos  en el
funcionamiento  de   todo  el   sistema  en  su   conjunto,  subgrupos
relacionados,  relaciones inesperadas,  entre otras.  Utilizando dicha
teoría se pretende analizar el proyecto colaborativo de software libre
Debian  y su  sistema operativo  Debian GNU/Linux  con el  objetivo de
encontrar  los factores  que inciden  en el  tiempo de  lanzamiento de
versiones estables  de este.  Para  lograr este objetivo  se manejaran
varios enfoques  para medir dicho  impacto y podemos  definirlos como:
utilizar como principales nodos la unidad mínima funcional de software
en  el proyecto  denominada  paquete, la  figura  de desarrollador  de
software siendo  en este  caso tanto un  desarrollador Debian  como un
desarrollador original  del software  que da  origen al  paquete, como
relaciones  entre   nodos  se  tomaran  en   cuenta,  dependencias  de
construcción entre  paquetes, dependencias  en tiempo de  ejecución de
los paquetes,  bugs detectados en  los paquetes que interfieran  en la
relación  entre ellos,  cantidad de  bugs  reportados de  parte de  un
desarrollador Debian  a un desarrollador upstrem,  cantidad de parches
aplicados por parte de upstream  enviados por un desarrollador Debian,
cantidad  de  correos  entre  desarrolladores.  Con  esta  información
modelada  se podrá  analizar  por medio  de  algoritmos existentes  de
teoría de análisis de redes sociales, en caso de poder ser aplicables,
de lo  contrario se creara el  adecuado para poder realizar  de manera
apropiada  los datos  y de  esta  forma poder  encontrar, patrones  de
centralidad,  incidencia  de  clanes,   nodos  de  mayor  importancia,
etcétera,  y con  esto  podrá  interpretarse los  valores  y hacer  un
análisis del impacto  en el tiempo de desarrollo de  una nueva versión
estable del proyecto de software a otra.
\todo{La intro explica bien de que se tratara el estudio, solo
que tenes que presentarla en un orden lógico}
\todo{Al haber decidido también tenes que explicar cual va a ser la 
metodología de investigación con la cual vas a abordar el problema
y la teoría computacional que la soporta}
\begin{comment}
Yo plantearia que es Debian, como funciona una distro a grandes rasgos
cuales son sus características que lo hacen un proyecto comunitario
orgánico, como se dan las conexiones, -i.e. todo lo que pusiste 
despues de hablar de SNA- y ya con esa información informar al lector 
que por tal motivo se usara SNA.

De nuevo también te recomiendo revisar y alternar conectores en todo
el texto.
\end{comment}


\chapter*{Antecedentes}  
\todo{Mismo comentario que la sección anterior, primero explica el
problema y luego como fue cubierto/estudiado}
\todo{También debes agregar una pequeña explicación de cada estudio
y/o su objetivo para que realmente sea una sección antecedentes, el 
lector no debe ir a cada link para entenderlo}
En  el  campo  de estudio  de  la  teoría  de
análisis  de redes,  se han  realizado investigaciones  relacionadas a
proyectos  de software  pudiendo  mencionar  análisis de  comunicación
entre  desarrolladores en  un proyecto  de software\footnote{Crowston,
Kevin, and  James Howison.   ``The social structure  of free  and open
source  software  development.''   First  Monday  10.2  (2005).},  las
implicaciones  de la  cordinacion de  la arquitectura  de software  en
proyectos    de     software\footnote{Ryan,    Sharon,     and    Rory
V. O\'Connor. ``Development  of a team measure for  tacit knowledge in
software  development teams.''  Journal of  Systems and  Software 82.2
(2009): 229-240.}, para encontrar  patrones de colaboracion en equipos
de  desarrollo de  software\footnote{Ehrlich, Kate,  Giuseppe Valetto,
and Mary Helander.  ``Seeing inside: Using social  network analysis to
understand  patterns  of  collaboration  and  coordination  in  global
software   teams.''   Global   Software   Engineering,   2007.   ICGSE
2007.  Second IEEE  International Conference  on. IEEE,  2007.}, entre
otros casos  de estudio en el  área de ingeniería de  software. De tal
forma que el estudio se encargara en el área de tiempos de lanzamiento
y  de  desarrollo   que  se  basan  en   dichos  estudios  anteriores,
aplicándolo a un proyecto colaborativo  de software libre y definiendo
un  modelo en  base a  los datos,  buscando encontrar  los principales
cuellos de botella y áreas a  mejorar para poder eficientar el proceso
de desarrollo de software de la distribución Debian GNU/Linux.

\chapter*{Justificación} 
\todo{Este es lenguaje de abogado, simplificalo y unilo a todo el
siguiente parrafo}
La investigación tiene un fundamento esencial
de existir debido a la necesidad  de poder encontrar los problemas que
se  llevan a  acabo en  proyectos  colaborativos de  software de  gran
envergadura.

\todo{Esta sección te puede ayudar a construir bien tu propósito
Propósito = Problema + Justificación simple + aporte a ciencia/área}
El motivo de la elección de  un proyecto comunitario de software libre
como Debian GNU/Linux se debe a factores sociales y técnicos, sociales
debido  a  que  la  estructura   organizacional  de  la  comunidad  de
desarrolladores es pública y abierta lo cual es una ventaja para poder
obtener información  de relaciones entre  toda la comunidad.  Desde el
punto de vista  técnico, al ser un proyecto que  desarrolla un sistema
operativo  de software  libre permite  poder  tener acceso  a toda  la
información referente a los enlaces y relaciones que existen entre las
unidades de software que forman al sistema operativo en general, desde
relaciones de dependencias directas  hasta relaciones indirectas en el
funcionamiento integral del sistema operativo como tal.
\todo{Aca se confunden dos cosas, tu estudio\newline
¿Pretende describir el problema especifico de Debian?
\newline o
\newline ¿Pretende describir el problema especifico de Debian, para 
aportar un estudio base que pueda servir al área de ingeniería de 
software?
\newline Ambas propuestas son validas pero no queda claro y tampoco
se menciona en el propósito, se menciona algo en los siguientes
parrafos pero están desconectados por el parrafo entre ambos.
}

Debido a lo anteriormente expuesto  tomando en cuenta otras teorías de
análisis  de   datos  existentes,   la  mas  adecuada   para  analizar
información  y poder  encontrar  factores medibles  que  afecten a  la
liberación de nuevas versiones de software es la teoría de análisis de
redes sociales  esto debido a  que la organización del  proyecto tanto
social  como técnicamente  pueden  ser descritos  como  una red  donde
pueden encontrarse  tendencias y poder aplicarse  un modelo replicable
en  otras   investigaciones,  el  proyecto  ya   provee  un  entramado
interrelacionado lo cual facilita la  utilización de análisis de redes
sociales en la organización.
\todo{De nuevo aca no hables de SNA, SNA es importante para resolver
el problema pero aca estas cubriendo porque tu problema es importante
y a quien/quienes puede ayudar trabajar sobre el}
\todo{Si hablas de SNA tenes que listar al menos cuales otras técnicas
exploraste como opción.}

El proyecto es beneficioso por  que brindará un modelo replicable para
todo gestor  de software que  cuente con  un proyecto de  software con
múltiples componentes  interrelacionados y un equipo  de colaboradores
heterogéneos tanto internos(desarrolladores) como externos(terceros).
\todo{De nuevo, aca no hables de SNA, SNA es importante para resolver
el problema pero aca estas cubriendo porque tu problema es importante
y a quien/quienes puede ayudar trabajar sobre el}

Las implicaciones que tendrá el trabajo irá orientado a la creación de
un modelo replicable para determinar tiempos de lanzamientos.
\todo{¿Porque replicable?, sin evidencia es dificil de afirmar que
en efecto lo sera}

El trabajo va dirigido a toda  persona que necesite modelos para poder
determinar que puntos modificar en su organización de manera técnica y
social  de  tal forma  que  pueda  mejorar  tiempos de  lanzamiento  y
obteniendo calidad en un tiempo razonable con una mayor anticipación.

\chapter*{MARCO TEÓRICO}
\todo{De nuevo, presenta primero el problema sus características,
luego las características deben justificar solas la elección de SNA,
presentando brevemente otras opciones}
1.Análisis de redes  sociales \\ 
1.2 Centralidad,  cuellos de botella,\\ 
1.3  Clicas, clusters y  componentes, \\ 
2.Proyecto Debian,  \\ 
2.1 Estructura  social  \\  
2.2  Estructura   técnica  \\  
2.3  Medios  de comunicación \\
\todo{El marco teorico tambien debe cubrir la teoría computacional de
lo que vas a utilizar para tu estudio, incluyendo: métodos de 
recolección de datos, metodos de análisis, alternativas estudiadas}
%En  el caso  del  protocolo, el  marco teórico  funge  como un  marco
%conceptual,   indicando    ciertas   definiciones    conceptuales   y
%explicaciones   breves   con   que   se   basará   el   proyecto   de
%investigación. Para la entrega del  protocolo esta sección debe tener
%aproximadamente de 2 a 3 páginas de conceptualización.

\end{document}

%TODO                                                       referencia
http://www.researchgate.net/publication/271625890_Analysis_of_activity_in_open-source_communities_using_social_network_analysis_techniques
%TODO http://www3.nd.edu/~oss/Papers/HICSS40_final.pdf
