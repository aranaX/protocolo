\documentclass[12pt]{report}
\usepackage[utf8]{inputenc}
\usepackage[spanish]{babel}
\usepackage{hyperref}
\usepackage{verbatim}
\usepackage{xargs}
\usepackage{changebar}
\usepackage[colorinlistoftodos, textwidth=65mm, shadow]{todonotes}
\usepackage[paperwidth=275.9mm, paperheight=279.4mm]{geometry}
\setlength{\oddsidemargin}{35mm}
\setlength{\evensidemargin}{35mm}
\setlength{\voffset}{-1in}
\setlength{\hoffset}{-1in}
\setlength{\textwidth}{156mm}
\setlength{\topmargin}{4mm}
\setlength{\headheight}{10mm}
\setlength{\headsep}{12mm}
\setlength{\topskip}{0mm}
\setlength{\textheight}{228mm}

\begin{document}  \title{ {Análisis  de  tiempos  de  lanzamiento  de
proyectos comunitarios  de software  libre mediante análisis  de redes
sociales:   Caso   practico    Debian   GNU/Linux}
\\   {Protocolo}   }
\author{Wences Arana}
\maketitle

\tableofcontents
\chapter*{Propósito del proyecto}

Todo proyecto de software inicia como  una idea general hasta llegar a
ser un sistema íntegro y  funcional, para lograrlo conlleva un proceso
ordenado con límites definidos en  cuestiones de tiempo y cumplimiento
de metas.   Dicho cumplimiento de  metas debe  ser dentro de  un marco
tolerable de tiempo, esto con el  fin de tener un proceso eficiente de
desarrollo  de software,  un  mejor manejo  de  costos de  producción,
calidad del  mismo entre  otros. Aunque  varios proyectos  de software
libre como en el caso del Proyecto Debian no cuentan con dicha rigidez
de tiempo si  se desea que el software obtenido  para lanzar de manera
estable deba contar con un estándar de calidad alto.

Por esto mismo  la medición de los factores que  influyen en el atraso
del cumplimiento de metas se hace de vital importancia para el área de
la  ingeniería de  software teniendo  como finalidad  lograr crear  un
proceso eficiente  de desarrollo replicable  y dentro de  los recursos
disponibles  o   proyectados.  Debemos   por  lo  tanto   plantear  la
realización  de un  análisis  por medio  de un  estudio  de los  datos
disponibles y crear un modelo  replicable en proyectos de software con
características similares, en nuestro caso  se hará el estudio en base
a sistemas operativos libres  desarrollados de manera colaborativa por
voluntarios y  otros actores  interesados, tomando en  cuenta factores
relacionados con la  unidad mínima de software  denominada paquete que
conforma una red de estos que  a su vez conforman el sistema operativo
en su  totalidad, además de  las redes  que existen en  las relaciones
sociales de los  participantes en el desarrollo de  dicho software. Se
utilizará como base de este estudio  a el Proyecto Debian debido a que
provee  esta  infraestructura,  además de  tener  actualmente  algunos
problemas de retraso  en entregas de lanzamientos de  su rama estable,
dándonos el escenario ideal para realizar el estudio.

Poder predecir los tiempos de  entrega, ayudará a proyectos grandes de
software  en la  economización de  recursos, debido  a la  detección a
tiempo de cuellos de botella  en desarrollo, colaboradores críticos en
el  proyecto, mala  comunicación  entre los  distintos equipos,  entre
otros.  Ayudando  así en  la  gestión  del  proyecto  con la  toma  de
decisiones relacionadas  con los  componentes que  forman parte  de su
proyecto de  software tanto lo social  como lo técnico, lo  tangible e
intangible.

\chapter*{Introducción} En  el área de administración  de software una
de las principales  áreas de estudio es la entrega  de lanzamientos en
tiempo  y  utilizando los  recursos  disponibles,  si el  proyecto  de
software   no  es   entregado  en   el  tiempo   estipulado  impactará
directamente en el presupuesto del proyecto, la confianza en el equipo
de  desarrollo,   y  afectará   la  confianza   en  el   entregable  a
utilizar. Poder  predecir a  tiempo los puntos  que puedan  atrasar el
proyecto ayudaría a entregar el software sin impactar negativamente en
los puntos mencionados  anteriormente. En el caso  del proyecto Debian
GNU/Linux  dicha problemática  se ha  visto evidenciada  a través  del
tiempo,  a  pesar de  no  tener  un  esquema de  tiempo  estrictamente
definido,  se intenta  lanzar una  nueva versión  estable del  sistema
operativo en  el momento en  que ciertas  metas se cumplan,  muchas de
estas no se llegan a lograr en un tiempo lo suficientemente rápido, de
tal forma que  se distribuya software más antiguo del  que se presenta
en otras distribuciones GNU/Linux.

Para  poder  llegar  a  determinar   las  causas  que  afectan  en  el
lanzamiento  de tiempo  se decide  utilizar la  teoría de  análisis de
redes, esto debido  a que tiene como objetivo  encontrar relaciones en
eventos  de la  naturaleza que  no parecen  tener incidencia  unos con
otros a primera vista, sin embargo al realizar un análisis profundo de
sus relaciones, en algunos  casos, pueden encontrarse desde relaciones
de causa efecto entre eventos, actores, recursos, hasta la importancia
de algunos  de estos  en el  funcionamiento de todo  el sistema  en su
conjunto, subgrupos relacionados, relaciones inesperadas, entre otras.

Utilizando dicha teoría se  pretende analizar el proyecto colaborativo
de software libre  Debian y su sistema operativo  Debian GNU/Linux con
el objetivo  de encontrar  los factores  que inciden  en el  tiempo de
lanzamiento de versiones estables de  este.  Para lograr este objetivo
se  manejaran  varios enfoques  para  medir  dicho impacto  y  podemos
definirlos  como: utilizar  como  principales nodos  la unidad  mínima
funcional de software en el  proyecto denominada paquete, la figura de
desarrollador de software  siendo en este caso  tanto un desarrollador
Debian como  un desarrollador original  del software que da  origen al
paquete,   como  relaciones   entre  nodos   se  tomaran   en  cuenta,
dependencias de construcción entre paquetes, dependencias en tiempo de
ejecución  de  los  paquetes,  bugs detectados  en  los  paquetes  que
interfieran en la relación entre ellos, cantidad de bugs reportados de
parte de un desarrollador Debian a un desarrollador upstream, cantidad
de  parches   aplicados  por  parte   de  upstream  enviados   por  un
desarrollador Debian, cantidad de  correos entre desarrolladores.  Con
esta información  modelada se podrá  analizar por medio  de algoritmos
existentes de teoría  de análisis de redes sociales, en  caso de poder
ser  aplicables, de  lo contrario  se  creara el  adecuado para  poder
realizar  de  manera  apropiada  los  datos  y  de  esta  forma  poder
encontrar,  patrones de  centralidad, incidencia  de clanes,  nodos de
mayor importancia,  etcétera, y  pudiéndose interpretar los  valores y
hacer un análisis del impacto en  el tiempo de desarrollo de una nueva
versión estable  del proyecto de  software a otra.  Se  utilizará como
metodología de  investigación la propuesta  por el muestreo  teórico o
``Grounded Theory'',  bastante utilizada  cuando se  realizan estudios
sociales, como  es en  teste caso  además de ser  un bien  definido de
actores conocidos.

\chapter*{Antecedentes} Los proyectos de  software libre tienen por lo
general  estructuras  colaborativas  comunitarias,  se  han  realizado
estudios  acerca  de  los  patrones  de  comunicación,  debido  a  que
investigar las estructuras sociales es  una forma útil de entender las
practicas  de  los  equipos  realizándolo por  medio  de  realizar  un
análisis  de  comunicación entre  desarrolladores  en  un proyecto  de
software utilizando  análisis de redes  sociales de los  correos entre
los  colaboradores que  resuelven  bugs\footnote{Crowston, Kevin,  and
  James  Howison.  ``The  social  structure of  free  and open  source
  software development.''  First Monday  10.2 (2005).}.  Es importante
además  de que  forma se  transmite  el conocimiento  entre todos  los
miembros del grupo se ha podido determinar por medio de la utilización
de la misma técnica que el conocimiento tácito del proyecto se lleva a
cabo de persona  a persona de manera directa o  por un intermediario en
posición  dominante de  tal forma  que pueden  determinarse flujos  de
comunicación   entre  personas   \footnote{Ryan,   Sharon,  and   Rory
  V. O\'Connor. ``Development of a team measure for tacit knowledge in
  software development teams.''  Journal  of Systems and Software 82.2
  (2009): 229-240.}.  Otros autores han tocado  el tema de la forma en
que la comunicación  fluye entre todo el equipo  de desarrollo, además
de  mapearlo  a   los  artefactos  de  software   que  estos  programan
\footnote{Ehrlich,   Kate,  Giuseppe   Valetto,  and   Mary  Helander.
  ``Seeing  inside:  Using  social   network  analysis  to  understand
  patterns  of  collaboration  and  coordination  in  global  software
  teams.''  Global  Software Engineering,  2007.  ICGSE  2007.  Second
  IEEE  International  Conference on.  IEEE,  2007.}.   Como se  puede
observar ya  existen estudios que  han abordado el tema  utilizando la
técnica  de SNA  aplicada  a comunidades  desarrolladoras de  software,
desde el punto de vista  de comunicación, colaboración además de poder
relacionarlas  con el  software en  sí. Con  estos estudios  como base
podremos  modelar de  una  mejor  forma nuestro  estudio  y siendo  el
elemento  diferenciador, poder  relacionar  los  resultados al  impacto
directo  con  el  tiempo  de   lanzamiento  del  sistema  operativo  y
relacionándolo con la antigüedad del software incluido dentro de este,
y   encontrando   además   los   cuellos  de   botella   y   problemas
específicamente para  nuestro caso  de estudio la  distribución Debian
GNU/Linux.

\chapter*{Justificación} En  la actualidad proyectos  colaborativos de
software tienen problemas al tratar de poder entregar nuevas versiones
estables de  sus programas,  debido a múltiples  razones, tanto  en la
parte  técnica   de  elección   de  herramientas  y   metodologías  de
desarrollo, así como en su parte  social, respecto a la relación entre
todos  los   desarrolladores  que  forman  parte   del  proyecto.   La
investigación busca  poder encontrar  las principales causas  de estos
retrasos por medio  de un análisis en el aspecto  social respecto a la
comunicación de los desarrolladores, y las relaciones existentes entre
los paquetes de software que conforman a todo el sistema operativo que
produce un proyecto colaborativo de software libre.

El  estudio  brindará  una  forma  de poder  medir  por  medio  de  lo
anteriormente expuesto  formas de poder encontrar  factores que pueden
retrasar  el lanzamiento  de  un software,  para  poder detectarlos  a
tiempo y poder enfocarse en la solución de dichos inconvenientes.

Se ha decidido  elegir un proyecto comunitario de  software libre como
Debian  GNU/Linux debido a factores sociales  y técnicos,  sociales
por que  la  estructura   organizacional  de  la  comunidad  de
desarrolladores es pública y abierta lo cual es una ventaja para poder
obtener información de  relaciones entre toda la  comunidad.  Desde el
punto de vista  técnico, al ser un proyecto que  desarrolla un sistema
operativo  de software  libre permite  poder  tener acceso  a toda  la
información referente a los enlaces y relaciones que existen entre las
unidades de software que forman al sistema operativo en general, desde
relaciones de dependencias directas  hasta relaciones indirectas en el
funcionamiento integral del sistema operativo. Utilizando al proyecto 
Debian podremos describir los problemas específicos que tiene de tal 
forma que pueda servir al área de ingeniería de software.

El trabajo va dirigido a toda  persona que necesite modelos para poder
determinar que puntos modificar en su organización de manera técnica y
social  de  tal forma  que  pueda  mejorar  tiempos de  lanzamiento  y
obteniendo calidad en un tiempo razonable con una mayor anticipación.

\chapter*{MARCO TEÓRICO}
\begin{enumerate}
\item Ingeniería de software
  \begin{enumerate}
  \item Definición
  \item Recurso humano
  \item Recurso de software
  \item Etapas del proceso de desarrollo de software
  \item Ciclo de vida de desarrollo
  \item Gestión del proyecto
  \item Aseguramiento de calidad    
  \item Tiempos de lanzamiento
  \item Estimación de tiempos y metas
  \end{enumerate}
\item Proyectos colaborativos de software libre
  \begin{enumerate}
  \item Definición
  \item Proyecto Debian
  \item Estructura organizacional
  \item Infraestructura técnica
  \item Versionamiento del sistema operativo Debian GNU/Linux
  \item Modelo de lanzamiento de distribución estable
  \item Problemas relacionados a lanzamientos de versiones estables
  \end{enumerate}
\item Análisis de redes sociales
  \begin{enumerate}
  \item Definición
  \item Centralidad, cuellos de botella
  \item Clicas, clusters y componentes
  \item Triadas
  \end{enumerate}
\item Caso de estudio
  \begin{enumerate}
  \item Clasificación de información
  \item Selección de información
  \item Métodos de recolección de datos
  \item Recolección de información
  \item Métodos de análisis  
  \end{enumerate}
\item Análisis de resultado
  \begin{enumerate}
  \item Técnicas de presentación de información
  \item Gráficas
  \item Resultados
  \item Notas Finales
  \end{enumerate}
\end{enumerate}

%En  el caso  del  protocolo, el  marco teórico  funge  como un  marco
%conceptual,   indicando    ciertas   definiciones    conceptuales   y
%explicaciones   breves   con   que   se   basará   el   proyecto   de
%investigación. Para la entrega del  protocolo esta sección debe tener
%aproximadamente de 2 a 3 páginas de conceptualización.

\end{document}

%TODO                                                       referencia
http://www.researchgate.net/publication/271625890_Analysis_of_activity_in_open-source_communities_using_social_network_analysis_techniques
%TODO http://www3.nd.edu/~oss/Papers/HICSS40_final.pdf
