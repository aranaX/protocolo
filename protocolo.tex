\documentclass[letterpaper,12pt,titlepage]{report}
\usepackage[utf8]{inputenc}
\usepackage[spanish=nohyphenation]{hyphsubst}
\usepackage[spanish]{babel}
\usepackage{hyperref}
\usepackage{verbatim}
\usepackage{xargs}
\usepackage{changebar}
\usepackage{pgfgantt}
%\usepackage{pdfpages}
%\usepackage[colorinlistoftodos, textwidth=65mm, shadow]{todonotes}
%\usepackage[paperwidth=275.9mm, paperheight=279.4mm]{geometry}
%\setlength{\oddsidemargin}{35mm}
%\setlength{\evensidemargin}{35mm}
%\setlength{\voffset}{-1in}
%\setlength{\hoffset}{-1in}
%\setlength{\textwidth}{156mm}
%\setlength{\topmargin}{4mm}
%\setlength{\headheight}{10mm}
%\setlength{\headsep}{12mm}
%\setlength{\topskip}{0mm}
%\setlength{\textheight}{228mm}
\begin{document}

\linespread{1.2}
%margen derecho reducido %Ver: http://www.artofproblemsolving.com/Wiki/index.php/LaTeX:Layout                  
\setlength{\textwidth}{430pt}                                                         


\title{{Factores que inciden en el retraso de  tiempos de lanzamiento de
    proyectos comunitarios de software libre mediante el uso  análisis de
    redes sociales: Caso practico Debian GNU/Linux}  {Protocolo}}
\author{Wences Arana}
\maketitle
\tableofcontents
\renewcommand{\chaptername}{}%Borra palabra capítulo
\renewcommand{\thechapter}{}%Borra número de capítulo

% \chapter*{Resumen}
% Se puede crear a partir de la introducción, en el protocolo no es
% necesario que incluya una introducción detallada
\setcounter{section}{0} \cleardoublepage

\chapter{Propósito del proyecto}

% TODO Porque se hace
% \chapter*{Propósito del proyecto} %Situación actual
Los proyectos de software se definen como procesos de transformación
de ideas hacia herramientas tecnológicas que resuelven necesidades de
los usuarios. Y para llevar a cabo estos proyectos, las metodologías
de gestión de proyectos sugieren la implementación de procesos
ordenados con límites y objetivos definidos, con el fin de obtener
ciclos de desarrollo y lanzamientos de Software eficientes y
predecibles.  No obstante, en la practica proyectos de Software Libre
han demostrado que la implementación de estos procesos no es una tarea
trivial, dada su naturaleza orgánica y colaborativa.

% Problematica
Considerando que la medición de los factores que
influyen en el atraso del cumplimiento de metas es de vital
importancia para el área de Ingeniería de Software, y dada la
inexistencia de un marco de referencia acerca de estos factores en el
área de de Software Libre, se hace evidente la necesidad de determinar
la existencia (si la hubiera) de factores que afectan los ciclos de
lanzamiento.

Divisando la resolución de este problema, se propone el estudio de un
proyecto de software existente, específicamente sistemas operativos
libres desarrollados de manera colaborativa por voluntarios y otros
actores interesados, siendo el elegido el proyecto Debian.

% Detalles del estudio
El proyecto Debian tiene como una de sus metas
lanzar una nueva versión estable del sistema operativo en el momento
en que ciertas metas se cumplen.  Sin embargo, muchas de estas no se
logran en un tiempo lo suficientemente rápido, teniendo como
consecuencia la distribución de paquetes software más antiguos en
relación a otras distribuciones GNU/Linux.

Poder predecir los tiempos de entrega, ayudará a proyectos grandes de
software en la economización de recursos, debido a la detección a
tiempo de cuellos de botella en desarrollo, colaboradores críticos en
el proyecto, mala comunicación entre los distintos equipos, entre
otros.  Ayudando así en la gestión del proyecto con la toma de
decisiones relacionadas con los componentes que forman parte de su
proyecto de software tanto lo social como lo técnico, lo tangible e
intangible.

\chapter{Introducción} En el área de administración de software una de
las principales áreas de estudio es la entrega de lanzamientos en
tiempo utilizando los recursos disponibles, si el proyecto de software
no es entregado en el tiempo estipulado impactará directamente en el
presupuesto del proyecto, la confianza en el equipo de desarrollo, y
afectará la calidad en el entregable a utilizar. Poder predecir a
tiempo los puntos que puedan atrasar el proyecto ayuda a entregar
software sin impactar negativamente en los puntos mencionados
anteriormente.  En el caso del proyecto Debian GNU/Linux dicha
problemática se ha visto evidenciada desde los inicios del proyecto, a
pesar de no tener un esquema de tiempo estrictamente definido, se
intenta lanzar una nueva versión estable del sistema operativo en el
momento en que ciertas metas se cumplan, muchas de estas no se llegan
a cumplir en un tiempo lo suficientemente rápido, causando que se
distribuya software bastante estable pero más antiguo del que se
presenta en otras distribuciones GNU/Linux.

Los proyectos de desarrollo de software libre tienen un componente
humano colaborativo, el cual es construido a través de las
aportaciones de código de cada uno de los desarrolladores, utilizando
medios de comunicación como listas de correo y sistemas de registro de
errores. Como puede notarse dichas relaciones dilucidan una red
interna de comunicación entre cada uno de los desarrolladores, dichas
redes puede ser estudiadas por medio del uso de técnicas provenientes
de la teoría del análisis de redes sociales, pudiendo utilizar
algoritmos y análisis de información que dichas teorías proveen para
modelar de una mejor forma la red y poder clasificar la información
determinando variables y factores que pueden apoyar nuestro estudio,
unido a esto también podemos considerar el usos de teoría de redes
sociales debido a que tiene como objetivo encontrar relaciones en
eventos del proyecto que no parecen tener incidencia unos con otros a
primera vista, sin embargo al realizar un análisis profundo de estas,
en algunos casos, pueden encontrarse desde relaciones de causa efecto
entre eventos, actores, recursos, hasta la importancia de algunos de
estos en el funcionamiento de todo el sistema en su conjunto,
subgrupos relacionados, relaciones inesperadas, entre otras.

Utilizando dicha teoría se pretende analizar el proyecto colaborativo
de software libre Debian y su sistema operativo Debian GNU/Linux con
el objetivo de encontrar los factores que inciden en el tiempo de
lanzamiento de versiones estables de este.  Para lograr este objetivo
se manejaran varios enfoques para medir dicho impacto y podemos
definirlos como: utilizar como principales nodos la unidad mínima
funcional de software en el proyecto denominada paquete, la figura de
desarrollador de software siendo en este caso tanto un desarrollador
Debian como un desarrollador original del software que da origen al
paquete, como relaciones entre nodos se tomaran en cuenta,
dependencias de construcción entre paquetes, dependencias en tiempo de
ejecución de los paquetes, bugs detectados en los paquetes que
interfieran en la relación entre ellos, cantidad de bugs reportados de
parte de un desarrollador Debian a un desarrollador del software que
da origen al paquete, cantidad de parches aplicados por parte de los
desarrolladores originales del software enviados por un desarrollador
Debian, cantidad de correos entre desarrolladores.  Con esta
información modelada se podrá analizar por medio de algoritmos
existentes de teoría de análisis de redes sociales, en caso de poder
ser aplicables, de lo contrario se creara el adecuado para poder
realizar de manera apropiada los datos y de esta forma poder
encontrar, patrones de centralidad, incidencia de clanes, nodos de
mayor importancia, etcétera, y pudiéndose interpretar los valores y
hacer un análisis del impacto en el tiempo de desarrollo de una nueva
versión estable del proyecto de software a otra.  Se utilizará como
metodología de investigación por medio de un diseño no experimental
ex post facto de tipo descriptivo, debido a que se hará un ánalisis
de información de hechos y no se experimentará directamente con ellos.

\chapter{Antecedentes}
Los proyectos de software libre tienen por lo general estructuras
colaborativas comunitarias\cite{raymond1999the}, ,se han realizado
estudios acerca de los patrones de comunicación, debido a que
investigar las estructuras sociales es una forma útil de entender las
practicas de los equipos realizándolo por medio de realizar un
análisis de comunicación entre desarrolladores en un proyecto de
software utilizando análisis de redes sociales de los correos entre
los colaboradores que resuelven bugs\cite{crowston2005social}.  Es
importante además de que forma se transmite el conocimiento entre
todos los miembros del grupo se ha podido determinar por medio de la
utilización de la misma técnica que el conocimiento tácito del
proyecto se lleva a cabo de persona a persona de manera directa o por
un intermediario en posición dominante de tal forma que pueden
determinarse flujos de comunicación entre
personas\cite{ryan2009development}.Otros autores han tocado el tema de
la forma en que la comunicación fluye entre todo el equipo de
desarrollo, además de mapearlo a los artefactos de software que estos
programan \cite{ehrlich2007seeing}.  Como se puede observar ya existen
estudios que han abordado el tema utilizando la técnica de SNA
aplicada a comunidades desarrolladoras de software, desde el punto de
vista de comunicación, colaboración además de poder relacionarlas con
el software en sí. Con estos estudios como base podremos modelar de
una mejor forma nuestro estudio y siendo el elemento diferenciador,
poder relacionar los resultados al impacto directo con el tiempo de
lanzamiento del sistema operativo y relacionándolo con la antigüedad
del software incluido dentro de este, y encontrando además los cuellos
de botella y problemas específicamente para nuestro caso de estudio la
distribución Debian GNU/Linux.

\begin{comment}
La información en este párrafo esta bien, mi único comentario es la
forma en la que lo escribís es un poco "al machetazo", si observas los
tres papers que referenciaste siempre van en este orden al escribir
los antecedentes: - Hechos generales aceptados en el área de
conocimiento que serán de ayuda para afirmar su teoría -e.g. que la
interacción de los desarrolladores en proyectos distribuidos es un
proceso social- - Hechos generales que hay en trabajos relacionados
-e.g. que hay estudios de proyectos de software libre como procesos
colaborativos- - Justifican el uso de la(s) técnica(s) de sus papers y
como es diferente a los trabajos relacionados -e.g. El trabajo de
Wolf(2011) demostró que se puede utilizar redes bayesianas para
predecir la influencia a futuro que tendra el retraso de X o Y
paquete, sin embargo desconsidera la relación inmediata por lo que, en
este trabajo se pretende utilizar SNA en linea con lo presentado con
los autores [98] y [99]-

Podes buscar más papers que tengan sección "Related Works" o en las
Introducciones para ver como se hace y como van de lo generalistico a
lo especifico y relacionado.
\end{comment}

\chapter{Justificación}
En la actualidad proyectos colaborativos de software tienen problemas
al tratar de poder entregar nuevas versiones estables de sus
programas, debido a múltiples razones, tanto en la parte técnica de
elección de herramientas y metodologías de desarrollo, así como en su
parte social, respecto a la relación entre todos los desarrolladores
que forman parte del proyecto.  La investigación busca poder encontrar
las principales causas de estos retrasos por medio de un análisis en
el aspecto social respecto a la comunicación de los desarrolladores, y
las relaciones existentes entre los paquetes de software que conforman
a todo el sistema operativo que produce un proyecto colaborativo de
software libre.

El estudio brindará una forma de poder medir por medio de lo
anteriormente expuesto formas de poder encontrar factores que pueden
retrasar el lanzamiento de un software, para poder detectarlos a
tiempo y poder enfocarse en la solución de dichos inconvenientes.

Se ha decidido elegir un proyecto comunitario de software libre como
Debian GNU/Linux debido a factores sociales y técnicos, sociales por
que la estructura organizacional de la comunidad de desarrolladores es
pública y abierta lo cual es una ventaja para poder obtener
información de relaciones entre toda la comunidad.  Desde el punto de
vista técnico, al ser un proyecto que desarrolla un sistema operativo
de software libre permite poder tener acceso a toda la información
referente a los enlaces y relaciones que existen entre las unidades de
software que forman al sistema operativo en general, desde relaciones
de dependencias directas hasta relaciones indirectas en el
funcionamiento integral del sistema operativo. Utilizando al proyecto
Debian podremos describir los problemas específicos que tiene de tal
forma que pueda servir al área de ingeniería de software.

El trabajo va dirigido a toda persona que necesite modelos para poder
determinar que puntos modificar en su organización de manera técnica y
social de tal forma que pueda mejorar tiempos de lanzamiento y
obteniendo calidad en un tiempo razonable con una mayor anticipación.

\chapter{Marco Teórico}
\begin{enumerate}
\item Ingeniería de software
  \begin{enumerate}
  \item Definición 
  \item Recurso humano
  \item Recurso de software
  \item Etapas del proceso de desarrollo de software
  \item Ciclo de vida de desarrollo
  \item Gestión del proyecto
  \item Aseguramiento de calidad
  \item Tiempos de lanzamiento
  \item Estimación de tiempos y metas
  \end{enumerate}
\item Proyectos colaborativos de software libre
  \begin{enumerate}
  \item Definición
  \item Proyecto Debian
  \item Estructura organizacional
  \item Infraestructura técnica
  \item Versionamiento del sistema operativo Debian GNU/Linux
  \item Modelo de lanzamiento de distribución estable
  \item Problemas relacionados a lanzamientos de versiones estables
  \end{enumerate}
\item Análisis de redes sociales
  \begin{enumerate}
  \item Definición
  \item Centralidad, cuellos de botella
  \item Clicas, clusters y componentes
  \item Triadas
  \end{enumerate}
\item Caso de estudio
  \begin{enumerate}
  \item Clasificación de información
  \item Selección de información
  \item Métodos de recolección de datos
  \item Recolección de información
  \item Métodos de análisis
  \end{enumerate}
\item Análisis de resultado
  \begin{enumerate}
  \item Técnicas de presentación de información
  \item Gráficas
  \item Resultados
  \item Notas Finales
  \end{enumerate}
\end{enumerate}

\chapter{Planteamiento del problema}
\section*{Alcances}
\begin{itemize}
%TODO Definir porque es importante cada alcance, para que se hace
\item Modelar una red de paquetes de software, relacionados por medio de las dependencias entre ellos, para analizar la importancia de cada uno
\item Modelar una red de colaboradores dentro del proyecto
  (desarrolladores Debian) y entre los colaboradores externos e
  internos (Desarrolladores Debian y desarrolladores originales de
  software)
\item Analizar las redes que pudieron determinarse y aplicar
  algoritmos de análisis de redes sociales
\item Analizar los datos y encontrar los paquetes con más errores
  críticos para lanzamiento (release critical) y determinar si dichos
  paquetes tienen un grado alto, en la red y si esto afecta en los
  tiempos de lanzamiento de la distribución
\item Analizar los datos referentes a las redes de colaboradores en el
  proyecto para determinar los colaboradores con más conexiones
\end{itemize}
\section*{Límites}
\begin{itemize}
\item Se utilizarán los algoritmos existentes de teoría de redes
  sociales para encontrar las características importantes de las
  redes, no se pretenderá en este estudio crear nuevos algoritmos de
  análisis
\end{itemize}

\chapter{Objetivos}
\section*{Objetivo General}
\begin{itemize}
\item Determinar los principales factores sociales y técnicos que
  influyen en el retraso de los ciclos de lanzamiento en un proyecto
  de software colaborativo, utilizando la teoría de análisis de redes
  sociales.
\end{itemize}

\section*{Objetivos Específicos}
\begin{itemize}
\item Analizar el grado de la interrelación entre los desarrolladores
  de la distribución Debian GNU/Linux para determinar los principales
  colaboradores del proyecto tomando como medida la cantidad de código
  realizado y errores detectados y/o corregidos.
\item Evaluar la importancia de los paquetes de software por medio de
  su grado en la red de paquetes en cuanto a sus dependencias y el
  impacto de tener un error crítico.
\item Enumerar los paquetes de la distribución respecto a su número de
  dependencias y su posibilidad de retrasar el lanzamiento.
\item Determinar los desarrolladores más activos dentro del proyecto,
  tomando como medida la cantidad de código aportado, bugs reportados,
  participación en lista de correo y mayor relación con otros
  desarrolladores.
\item Analizar el impacto de perder la colaboración de desarrolladores
  activos dentro del proyecto en el tiempo de lanzamiento de la
  distribución.
\item Examinar si la comodidad de las relaciones de colaboración entre
  los distintos desarrolladores que participan en la construcción de
  la distribución, tanto internos como externos impacta en el tiempo
  de lanzamiento de la distribución.
\item Analizar el efecto de un bug RC en distintos paquetes y
  determinar si existe una correlación entre un paquete importante de
  la distribución con este tipo de bugs y el tiempo de lanzamiento de
  la distribución.
\end{itemize}

% TODO Como se pretende determinar
%ver http://www.socscidiss.bham.ac.uk/methodologies.html
\chapter{Metodología de investigación} %TODO Expandir esta sección Se
La investigación será de tipo cuantitativo, por medio de un diseño 
no experimental ex post facto de tipo declarativo. Se decidió utilizar
este enfoque debido a que se hará una recolección de datos de sucesos
del pasado (correos electrónicos de colaboradores de la distribución, 
paquetes de software que conformaban a la distribución en un tiempo
determinado del pasado, cantidad de errores que poseia un paquete en
un periodo de tiempo determinado en el pasado). Por lo tanto no se puede
realizar un experimento debido a que los datos son de hechos que ya sucedieron
antes. Se opta por el modelo descriptivo y de análisis de relaciones de
influencia.

Para el análisis de los datos ex post facto se utilizará la metodología
de análisis de redes sociales, por la estructura de la información
que puede ser agrupada en forma de redes y aplicando algoritmos de 
dicha metodología facilitará la elaboración de factores de correlación,
que se acoplan a los buscados por nuestra investigación.

Se buscara describir dichos factores de correlación en el desarrollo de 
la distribución Debian GNU/Linux versión jessie en su periodo de desarrollo
en el periodo de años 2013-2015. 

\chapter{Recursos}
\begin{itemize}
\item Base de datos de errores del Proyecto Debian (Debian UDD)
\item Base de datos de los paquetes del Proyecto Debian
\item Base de datos de la lista de correo de errores y de los
  desarrolladores
\item Computadora con el poder de procesamiento para analizar más de
  30,000 paquetes interrelacionados
\item Acceso a información del proyecto
\item Investigador
\item Asesor
\item Al menos 2 desarrolladores Debian
\end{itemize}

% TODO En linea con los objetivos y el proposito
\chapter{Resultados esperados}
Poder determinar la existencia de redes entre los distintos
participantes de un proyecto colaborativo de software, desde el punto
de vista social, es decir, la parte humana del proyecto
(desarrolladores) y la parte técnica conformada por los paquetes. De
existir las redes poder utilizarlas para analizar y encontrar y
validar los posibles factores que afectan en el lanzamiento de una
nueva versión estable. De tal forma que se esperara obtener:

\begin{itemize}
\item Mapear una red de paquetes unidos por sus relaciones de dependencias
\item Mapear una red de colaboradores dentro del proyecto
\item Obtener en ambas redes valores de centralidad
\item Por medio de los datos obtenidos a partir del análisis de las
  redes, si es posible definir valores de centralidad, existencia de
  clusters, clicas, entre otros.
\item Determinar importancia de los paquetes por medio de análisis de centralidad
\item Determinar paquetes críticos en base a su cantidad de errores e importancia en la red
\item Determinar colaboradores mas activos en cuanto a errores reportados y corregidos
\item Determinar colaboradores mas activos respecto a su comunicación
  a través de los medios de comunicación del proyecto
\end{itemize}

\chapter{Asesor sugerido}

\href{http://vorozco.com/cv/index-es.html}{Ing.  Msc.  Víctor Leonel
  Orozco López}


\chapter{Cronograma de trabajo}
%\includepdf[pages={1}]{grantt.pdf}

%\begin {ganttchart}{1}{12}
  % \gantttitle {2016}{8} \\
  % \gantttitlelist {Enero, Febrero, Marzo, Abril, Mayo, Junio, Julio,
  % Agosto}{1} \\
  % \gantttitlelist {1,...,12}{1} \\
  % \ganttgroup {Elaboración de tesis}{1}{7} \\
  % \ganttbar {Marco teórico}{2}{3} \\
  % \ganttlinkedbar {Recolección de datos}{3}{5} \\
  % \ganttlinkedbar {Análisis de datos}{5}{7} \\
  % \ganttlinkedbar {Conclusiones}{7}{8}
%\end {ganttchart}



% TODO Gantt con entregables 1 mes y medio marco teórico 1 mes y medio
% obtención de datos 1 mes y medio análisis de datos 1 mes
% conclusiones


\bibliography{biblos}{} \bibliographystyle{plain}

\end{document}

%https://packages.debian.org/sid/dctrl-tools
%iter_paragraphs
% apt-cache rdepends <packagename> and scrape terminal output too
% apt-rdepends
% bottom of the page https://bugs.debian.org/cgi-bin/pkgreport.cgi?src=gtk+3.0
% <themill> quertybts is the command line utility to look things up; python-debianbts is the SOAP wrapper; Judd has a UDD wrapper.
% http://git.nanonanonano.net/?p=judd.git;a=tree;f=supybot/plugins/Judd/uddcache


%TODO                                                       referencia
%http://www.researchgate.net/publication/271625890_Analysis_of_activity_in_open-source_communities_using_social_network_analysis_techniques
%TODO http://www3.nd.edu/~oss/Papers/HICSS40_final.pdf
